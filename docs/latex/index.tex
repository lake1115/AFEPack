In fact, I make it not only for Electromagnetic calculation, but also for the general P\+DE cases.

In this note, the variational principles for electromagnetic is mainly based on Jian-\/\+Ming Jin\textquotesingle{}s the book “\+The Finite Element Method in Electromagnetics”.\hypertarget{index_nodeFEM}{}\section{Node Finite Elements}\label{index_nodeFEM}
Consider the P\+ED in a Lipschitz domain $ \Omega\in\mathcal{R}^d$, $ d=2,3$ \[ -\nabla \cdot (c(x)\nabla u) +a(x)u =f \qquad \in \Omega \] with right hand side $ f \in L^2(\Omega)^d $.

The boundary condition is the homogeneous Dirichlet boundary condition \[ u = b \qquad on \ \partial\Omega_1 \] and the homogeneous Neumann boundary condition \[ \hat{n}\cdot c(x)\nabla u + q(x)u = g \qquad on \ \partial\Omega_2 \] provided that a,c,q is real or complex and $ \hat{n}$is the outward normal from S, where $ \partial\Omega_1+\partial\Omega_2 = \partial\Omega $\hypertarget{index_forms}{}\subsection{Vartiational Formulation}\label{index_forms}
by applying the first scalar Green\textquotesingle{}s theorem \[ \int_\Omega v \nabla \cdot (c\nabla u) + c(\nabla v) \cdot (\nabla u) dV = \oint_{\partial\Omega} cv\frac{\partial u}{\partial n}dS \] ~\newline
and the boundary conditions above, we can get \[ \int_\Omega c(\nabla v) \cdot (\nabla u)dV +\int_\Omega a(x)uv dV = \int_\Omega fv dV + \oint_{\partial\Omega_2} v(g-q(x)u)dS \] the forms \[ a(u,v) := \int_\Omega c(\nabla v) \cdot (\nabla u) dV +\int_\Omega a(x)uv dV+\oint_{\partial\Omega_2}q(x)vudS \] \[ rhs(v) := \int_\Omega fv dV + \oint_{\partial\Omega_2} vgdS \]\hypertarget{index_exp1}{}\subsection{Example1}\label{index_exp1}
The computational domain is $[-1,1]\times [-1,1]$. The parameters \[ c(x) = 1 ,\quad a(x) = 0 \quad and \quad f = 2\pi^2\sin(\pi x)\sin(\pi y) \] the boundary parameters \[ b = 0, q(x) = 0 ,\quad and \quad g = \left\{ \begin{array}{ll} \pi \sin(\pi x) &\quad x \in [-1,1] \quad and \quad y=-1 \\ -\pi \sin(\pi y) &\quad x = 1 \quad and \quad y \in [-1,1] \\ -\pi \sin(\pi x) &\quad x \in [-1,1] \quad and \quad y=1 \\ \pi \sin(\pi y) &\quad x = -1 \quad and \quad y \in [-1,1] \end{array}\right. \] The exact solution is \[ u=\sin(\pi x)\sin(\pi y)\]

\href{../../example1.tiff}{\tt This figure} shows the results of Example.\hypertarget{index_vectorFEM}{}\section{Vector Finite Elements}\label{index_vectorFEM}
consider the vector-\/valued model problem \[ \nabla \times (c(x) \nabla \times \mathbf{u}) +a(x) \mathbf{u} = \mathbf{f} \qquad \in \Omega \] The boundary conditions are the homogenous Dirichlet boundary condition \[ \hat{n} \times \mathbf{u} = b \qquad on \ \partial\Omega_1 \] and the homogeneous Neumann boundary condition \[ c(x)\hat{n}\times (\nabla \times \mathbf{u}) + q(x)\hat{n}\times (\hat{n}\times \mathbf{u}) = \mathbf{g} \qquad on \ \partial\Omega_2 \] with $\partial\Omega_1+\partial\Omega_2=\partial\Omega$, a,c,and b,q are real(complex) numbers or functions.\hypertarget{index_forms}{}\subsection{Vartiational Formulation}\label{index_forms}
Invoking the first vector Green\textquotesingle{}s theorem and vector identity \[ \int_\Omega (c\nabla \times \mathbf{v}) \cdot (\nabla \times \mathbf{u}) - \mathbf{v} \cdot (\nabla \times (c\nabla \times \mathbf{u})) dV = \oint_{\partial \Omega} c(\mathbf{v}\times \nabla \times \mathbf{u}) \cdot \hat{n} dS \] \[ (\mathbf{v}\times \nabla \times\mathbf{u}) \cdot \hat{n} = (\hat{n}\times \mathbf{v}) \cdot (\nabla \times \mathbf{u})=-\mathbf{v}\cdot (\hat{n} \times (\nabla \times \mathbf{u})) \] applying the boundary conditions we obtain \[ \int_\Omega c(\nabla \times \mathbf{v})\cdot(\nabla\times \mathbf{u})dV+\int_\Omega a(x)\mathbf{u}\cdot\mathbf{v} dV= \int_\Omega \mathbf{f}\cdot\mathbf{v}dV-\oint_{\partial \Omega}\mathbf{v}\cdot(\mathbf{g}-q(x)\hat{n}\times(\hat{n}\times\mathbf{u}))dS \] the forms \[ a(\mathbf{u},\mathbf{v}) := \int_\Omega c(\nabla \times \mathbf{v})\cdot(\nabla\times \mathbf{u})dV+\int_\Omega a(x)\mathbf{u}\cdot\mathbf{v} dV+\oint_{\partial\Omega_2}q(x)(\hat{n}\times\mathbf{u})\cdot(\hat{n}\times\mathbf{v})dS \] \[ rhs(\mathbf{v}) := \int_\Omega \mathbf{f}\cdot\mathbf{v}dV-\oint_{\partial \Omega}\mathbf{v}\cdot \mathbf{g}dS \]\hypertarget{index_exp}{}\subsection{Example}\label{index_exp}
The computational domain is $[-1,1]\times [-1,1]$. The parameters \[ c(x) = 1 ,\quad a(x) = 1 \quad and \quad f = \left( \begin{array}{l} (2\pi^2+1)\cos(\pi x)\sin(\pi y)\\ -(2\pi^2+1)\sin(\pi x)\cos(\pi y) \end{array}\right) \] the boundary parameters \[ b = 0, q(x) = 0 ,\quad and \quad g = \left\{ \begin{array}{ll} (-2\pi \cos(\pi x),0) &\quad x \in [-1,1] \quad and \quad y=-1 \\ (0,-2\pi \cos(\pi y)) &\quad x = 1 \quad and \quad y \in [-1,1] \\ (2\pi \cos(\pi x),0) &\quad x \in [-1,1] \quad and \quad y=1 \\ (0,2\pi \cos(\pi y)) &\quad x = -1 \quad and \quad y \in [-1,1] \end{array}\right. \] The exact solution is \[ u=\left( \begin{array}{l} \cos(\pi x)\sin(\pi y)\\ -\sin(\pi x)\cos(\pi y) \end{array}\right) \]

\href{../../example2.tiff}{\tt This figure} shows the results of Example 